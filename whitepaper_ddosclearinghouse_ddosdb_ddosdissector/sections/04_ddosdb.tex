\section{The DDoSDB}
\label{sec:ddosdb}

More than a database with DDoS attack fingerprints.


Requirements:
\begin{itemize}
	\item flexible regarding the number and types of fields; 
	\item flexible access control
	\item distributed fashion;
	\item enrich the fingerprint;
	\item facilitate queries;
	\item enable download of fingerprints and anonymized data;
	\item facilitate notification;
	\item distribute information;
	\item upload information from only from trusted parties;
\end{itemize}

The aim of DDoSDB is twofold. On the one hand, it can help to create proactive measures of defying DDoS attacks, or recognise them in a very early stage. On the other hand, it can help us link the origin of multiple independent attacks together, thereby aiding law enforcement in the apprehending of the perpetrator. 

As stated in \ref{sec:ddos_dissector}, 