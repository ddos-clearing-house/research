\section{Introduction}
\label{introduction}

Distributed Denial of Service attacks exist since the beginning of the Internet. The goal of these types of attacks is to make target systems (services, devices, or even entire networks) unreachable to its intended users. Over time, these types of attacks increased in frequency and intensity. In 2011, the peak record was reported as 60Gb/s~\cite{}, in 2015 it was 500Gb/s~\cite{}, and in 2016 was 1.1Tb/s~\cite{}. As the dependency of our society to online service also increased, the damage cause by DDoS attacks has become extremely big.  While in 2015 large corporations reported the average loss of \$US410.000 per attack~\cite{}, in 2017 this figure increased to \$US2.500.000~\cite{}.

There are hundreds of companies worldwide offering DDoS protection and a large volume of academic work on DDoS attacks. Overall, DDoS attacks have been addressed as a reactive approach, waiting for attacks to hit a network infrastructure and see whether (i) the usual anomaly-based detection/mitigation solution works, or (ii) the network operators are skilled enough to mitigate the attack as fast as possible, or (iii) by simply paying more to third-party companies for having more network capacity and better protection. From the academic side, by the end of February 2019, Google scholar returned more than 47 thousand works on ‘ddos attack’. 

The question that we intend to discuss in this paper is \textbf{why and how (we believe), together, we can solve DDoS attacks?} The answer is technically simply and it is \textbf{not} by ``reinventing the wheel'', instead, facilitating filling the gaps of existing solutions and stakeholders (\eg victims, network operators, network security community, network security companies, law enforcement agencies, and the academic community). What we propose is a proactive approach that any stakeholder involved with a DDoS attack would benefit and trust.

Overall, we propose to facilitate: (1) victims, (2) network operators, and (3) network security companies to share their attack measurements (filtered and properly anonymized) and to get in return specific rules for detecting and mitigating those attack based on the already in-place solutions; (4) law enforcement agencies to compare attacks suffered in the society, for enabling legal attribution and prosecution of attackers (and buyers of attacks); (5) network security community (specially CERT/CSIRT) to get frequent feeds with IP addresses involved in attacks, towards preventing misused machines from performing attacks; and (6) the academic community on getting real/`fresh' DDoS attack data for testing and improving their solutions. For satisfying these six stakeholders, we propose and extensively validate the following three elements: 

\begin{itemize}
	\item \textbf{DDoS Dissector}: is a tool for analysing any type of network trace containing a DDoS attack (for example, pcap, pcapng, neflow, ipfix, sflow, and apache log), filter only the main characteristics of the attack, called \textbf{DDoS fingerprint}, and enable to share only the DDoS fingerprint and the anonymized version of the filtered attack. The requirements, the design and the validation of the tool is presented at \autoref{sec:ddos_dissector}; %100\% true positive
	
	\item\textbf{DDoS Fingerprint Converters}: is a set of tools for parsing the generic DDoS fingerprints into specific detection and mitigation technologies, for example, BGP Flowspec, eBPF, IPtables, SNORT, SURICATA, BRO, ModSecurity, and even ‘black-boxes’ from private security companies. An additional, and very important tools added to this set of converters, called \textbf{DDoS Mitigation Impact Quantification}, is responsible to validate and adapt detection and mitigation rules. The descriptions and explanations are presented at \autoref{sec:fingerprint_converters};
	
	\item \textbf{DDoS Database (DDoSDB)}: is a distributed database that receives, enriches, distributes, and make available: filtered anonymized attack traces, DDoS fingerprints, signatures/rules for specific hardware/software detecting/mitigating DDoS attacks, lessons learned from network operators, information from law enforcement agencies and feeds for CERT/CSIRT sanitize their networks. The requirements, the design and the validation of the tool is presented at \autoref{sec:ddosdb}.
\end{itemize}

After we describe the DDoS Dissector, the DDoS Fingerprint Converters, and the DDoS Database (DDoSDB), we introduce what we call as \textbf{DDoS Clearing House} in \autoref{sec:ddos_clearinghouse}. Only at that point we put all the pieces together and show how we have deployed \DDOSDBINSTANCES instances of the DDoSDB, collected more than \DDOSDBMULTIVECTORKEYS, and benefit more than \PROTECTEDORGANISATIONS organizations. The development of the tools are available at \url{https://github.com/ddos-clearing-house} and the public version of DDoSDB is available at \url{https://ddosdb.org}.


%Distributed Denial of Service (DDoS) attacks \citep{}
%
%Why standing on the shoulders of gigants: existing solutions and stakeholders. Not reinvent the wheel but adding 
%
%The solutions against DDoS attacks are based on technologies (hardware and software) that monitors the network traffic and react upon a predefined event. There are mainly two types of events. The first is when occurs a match between the incoming network traffic (packet or flow) and a predefined rule or signature, called rule/signature-based Intrusion Prevention System (IPS). The second type of event is when the network traffic has an abnormal behaviour based on a predefined profile. This latter type of technology is called anomaly-based IPS. 
%
%An ideal scenario for the mitigation of a DDoS attack would be a distributed solution with a r connected to an anomaly-based IPS.
%
%Different types of attacks requires different types of mitigation strategy.
%
%The difference between an intrusion detection systems (IDS) and an  is that the latter can launch an action after a detection event, while the former records an event after the detection. Sometimes these terms are interchangeable by the community.
%
%There are two types of IDS/IPS: signature and anomaly-based. While the signature-based IPS is 
%
%For us, the DEFINITION of a DDoS fingerprint is an useful comprehensive summary of the characteristics of a single vector of a DDoS attack extracted from a network measurement.
%
%*misunderstanding digital fingerprint, signature, rule, pattern, characteristics, profile
%
%Law enforcement agencies do not like to call \emph{digital fingerprint} because in their field a fingerprint leads to an unique individuo.
%
%THE PROBLEM can be summarized as: (1) a lack of standard for fingerprinting DDoS attacks; (2) lack of an updated database with fingerprints; (3) lack of functions to convert DDoS fingerprints into applicable detection and mitigation rules; 
%
%Rule-based mitigation solutions (\eg firewall, BGP flowspec,) are an particular case of signature-based IDS.
%
%Types of network measurements are: (1) packet-based (\ie pcap and pcapng), (2) (net)flow-based (\eg v5, v9, IPFIX), (3) sflow, and (3) log-based. 
%
%USAGE of the fingerprint: (1) improve existing detection and mitigation solutions, (2) attribution, (4) correlation among DDoS attacks and correclaation with other types of cyber threats, (5) accounting for attacks, and (6) notification to CSIRT/CERT's for cleaning misused machines, (3) facilitate the access to attacks traffic (usually for academic purpose)
%
%Detection and mitigation tools for DDoS attacks: (1) BGP Flowspec, (2) BRO, (3) SURICATA, (4) SNORT, (5) ModSecurity, (6) eBPF, (7) IPtables.
%
%Unicast addresses
%
%How much is needed to be collected (time, packets, flows)? 
%How much is the impact of a mitigation rule (considering the ongoing traffic)?
%How to indicate spoofed IP presence?
%
%% Enrichments: AS, geolocation, open ports, spoofed?, blacklisted?,
%
%We intend to solve the problem that DDoS attacks pose by ...
%
%DDoS attacks are not likely to stop happening!



